Mangroves are woody plants that live in the coastal intertidal zone.
They are used in flavouring agents, textiles, housing, and more. Their root systems
also serve as a habitat for many species of fish. 
The mangroves themselves are a key part of mangal ecosystems. Some flora and fauna associated
with mangroves are algae, sea grasses, saltmarsh plants, sponges, nematodes, prawns, shrimp,
crabs, dragonflies, bees, ants, shipworms, and fish. 
There is high genetic diversity among mangroves, owing to the necessity of adapting to their habitats \cite{biologyOfMangroves}. In recent years, mangrove populations have significantly
declined--especially in developing countries, where about 90\% of them are found \cite{kavanagh2007}.

There are various factors that affect the growth of mangroves, including tidal
inundation, salinity, competition between other mangroves, and storms. 

There is a need for a stochastic model that can predict the regenerative
behavior of multi-species mangrove forests in a fragmented
habitat. Such a model can help inform the decision-making process of environmental
conservation efforts. The authors improve upon work done by Salmo \& Juanico \cite{SalmoJuanico2015} and an agent-based implementation by Ang \& Mariano \cite{mangrovesAngMariano} in order to create such a model. This is a shorter and summarized
version for CS 175.