Since this is a summarized report that is more concerned with the inner workings
of the model and the applications of stochastic simulations, there will be no in-depth
discussion of the results of experiments. For more information, please read the full version. 

A sensitivity analysis of the model was conducted. Simulations were run for 5000 steps, varying the number of blocks in the block disturbance model and the allometric
parameters $\alpha$, $\beta$, and $\gamma$ in a simulation of 100 runs. It should be noted that when $\alpha$ or $\beta$ were varied, the native genera dominated the environment. When $\gamma$ was varied, the non-native (planted) genera dominated.

The dominant population at the end of each configuration was also recorded. \emph{Avicennia \& Sonneratia}, the native genera, dominated in most configurations, as shown the table below. N indicates that the native genera dominated and P indicates that the planted mangroves \emph{Rhizophora} dominated. The number in the parentheses indicate how many times the species dominated.

\resizebox{0.45 \textwidth}{!}{%
\begin{tabular}{|c|c|c|c|c|}
\hline 
Storm Strength & 0 & 1 & 2 & 3 \\ 
\hline 
Vary Alpha & N (90\%) & N (99\%) & N (90\%) & N (71\%) \\ 
\hline 
Vary Beta & N (93\%) & N (100\%) & N (97\%) & N (72\%) \\ 
\hline 
Vary Gamma & N (100\%) & N (98\%) & N (76\%) & P (79\%) \\ 
\hline 
All Constant & N (99\%) & N (98\%) & N (72\%) & P (74\%) \\ 
\hline 
\end{tabular}}

Note here that the planted species fare better when storms are \emph{stronger},
since the native species have more established trees and thus are more vulnerable to
storms. After such a storm has devastated the native population, the non-native species
has a chance to thrive.

Native genera were shown to fare consistently better
than \emph{Rhizophora} when all of these mangrove populations were mixed
in the planted site. They displayed stable or slowly increasing
population numbers, the random occurrence of powerful typhoons
notwithstanding. 

In terms of regeneration, when counting the number of mature trees before each storm
and comparing the successive population growth, the native mangroves take about 4-5 years
to recover and regain the same number of mature trees. Planted mangroves take about 2-4 years to regain the number of trees after storms.

It should be noted that as storms become
stronger, the recovery times tend to go down, because the number of trees decrease and 
successive regrowth requires less and less trees to regain those numbers. The faster recovery rate of the planted mangroves can
be attributed to the same phenomenon. Their small population numbers mean that they need only a small amount of time before reaching the pre-disaster tree population again. More experiments will be done, counting seedlings and saplings, and making more mature trees count for more, in order to account for this. This is similar to the method of looking at foliage cover (more mature trees contribute more foliage) and would allow us to take into consideration the amount of biomass that needs to be regrown before pronouncing a population as "recovered".