\documentclass[a4paper]{scl-article}
\usepackage{amsmath,amssymb,graphicx,array,multirow,float,multicol,breqn}
\usepackage{xwatermark,tikz}

%\newsavebox\mybox
%\savebox\mybox{\tikz[color=gray,opacity=0.2]\node{DRAFT};}
%\newwatermark[
%  allpages,
%  angle=45,
%  scale=6,
%  xpos=-20,
%  ypos=15
%]{\usebox\mybox}

\begin{document}

\title{A Stochastic Agent-based Model of Mangrove Forest Regrowth: A Summarized Report for CS 175}

\author{Vincent Paul Fiestada, Andrew Vince Lorbis\\Adviser: Vena Pearl Bongolan, Ph.D.}

\maketitle

\begin{multicols}{2}
\chapter{Introduction}
\label{cha:intro}

\section{Background}
Mangroves are trees and shrubs that live in the coastal intertidal zone in tropical and subtropical regions. The root system of mangrove forests serve as a habitat to various aquatic organisms. Mangroves are also used in small-scale industries such as in flavouring agents, textiles, housing and more.

Mangroves live in highly saline, anoxic environments. Their seedlings can float and take root once it finds ground. Some species are \emph{viviparous} and have seeds that develop while still attached to the parent plant, then break off afterwards. Some mangroves have been known to live for more than 100 years. However, most trees will never reach full maturity because of various factors affecting mangrove growth and mortality. These include tidal inundation, salinity, and competition between individuals.

Coastal woodlands are also very vulnerable to storms. In 2009, Typhoon "Emong", known internationally as typhoon "Chan-hom", dealt major damage to the mangrove forests of the Philippines. While there had been replanting efforts in some areas, the locals opted to plant non-native \emph{Rhizophora} instead of \emph{Avicennia and Sonneratia}, which are native to the region. The decisions made regarding those efforts have been put into question due to the suitability of certain species to the site and the effects on recovery speed.

\section{Problem Statement}
The aim of this study is to develop a model that accurately simulates the regenerative behaviour of a multi-specific mangrove forest in a fragmented habitat influenced by tidal inundation, salinity, inter- and intra-species competition, and storms.

The agent-based modelling paradigm is used in order to simulate this system. Such a paradigm can be used in cases of complex systems for which no straightforward systems of equations exist that can model the population dynamics as empirically observed. The authors improve upon a model for individual growth developed by Salmo \& Juanico \cite{SalmoJuanico2015} and an agent-based implementation by Ang \& Mariano \cite{mangrovesAngMariano}.

\section{Significance of the Study}
There is a need for a stochastic model that can predict the regenerative behaviour of single and multi-species mangrove forests in a fragmented habitat. Such a model can help inform the decision-making process of environmental conservation efforts. 

As previously stated, mangroves are incredibly important components of mangal ecosystems. They serve as habitat and nursing grounds for various aquatic organisms. They are also used in various small-scale industries. However, mangrove forests routinely sustain heavy damage from typhoons. Rehabilitation efforts can play a key role in ensuring that they recover as quickly as possible. This model can be used to evaluate hypothetical rehabilitation programs for coastal woodlands in an inexpensive manner. It can assist biologists and environmentalists in choosing which genera of mangroves to plant and how that choice will affect the speed of recovery.
\section{Model Overview, Design concepts, and Details}
The model aims to simulate the growth or regeneration of a
multi-specific mangrove forest in a fragmented habitat. In particular,
the \emph{Avicennia, Sonneratia,} and \emph{Rhizophora} species are
investigated in a virtual analogue of the Bangrin Marine Protected Area in
Bani, Pangasinan. The virtual environment is a closed patch grid consisting of
some land area and sea area.

The main agent being modelled is the mangrove plant. These mangrove agents
have a single state variable that determines their growth and maturity:
the diameter at breast height of the mangrove, which will be referred to
as \(D\). Empirical data can be used to allometrically relate $D$ to other
factors such as tree height, foliage cover, and biomass \cite{Hiebeler2000}.

Agents can be classified in two ways: by species and by maturity. As the
model deals with multi-specific forests, the behaviour of mangrove
agents is modified by species-specific parameters. Mangroves are also
divided into maturity levels. In Ang and Mariano \cite{marovesAngMariano}, there were
three maturity levels (dependent on $D$) ranging from seedling to tree,
with different mortality rates tied to each maturity level per species. In
the present model, we obtain a Lagrange interpolated mortality rate from
the values used in \cite{mangrovesAngMariano}, so that a plant's chances of dying from natural causes gradually but steadily declines as its $D$ increases.

Salinity and tidal inundation responses are approximately dependent on
distance from the coast. The farther from the coast a patch is, the less
inundated and the less saline it should be due to having less exposure
to seawater. Each patch thus has a specific inundation and salinity effect to the
growth of the mangrove plant on it.

The model's determination of individual mangrove growth follows the
differential equation proposed by Salmo and Juanico. The deterministic change in $D$ with respect to time is as follows:

\begin{dmath}
\frac{\text{dD}}{\text{dt}} = (\frac{\Omega}{2 + \alpha})D^{\beta - \alpha - 1}\lbrack 1 - \frac{1}{\gamma}{(\frac{D}{D_{\max}})}^{1 + \alpha}\rbrack \times \sigma(x,y) \times \eta(x,y) \times K(x,y)
\end{dmath}


\(\alpha\)- allometric constant relating D to tree height

\(\beta\)- allometric constant relating D to crown radius

\(\Omega\)- conversion factor that harmonizes both sides of the equation

\(\gamma\)- species-specific constant related to maximum height

\(D_{\max}\)- species-specific constant indicating maximum attainable D

\(\sigma(x,y)\)- response to salinity

\(\eta(x,y)\)- response to inundation

\(K(x,y)\)- response to competition from nearby mangroves

Salmo and Juanico also provide formulas for calculating the responses to
salinity, tidal inundation and competition from patch values and
neighbourhood characteristics. The authors have left these unchanged.

The allometric parameters \(\alpha\), \(\beta\) and \(\gamma\) will be
species-specific but may also vary randomly among individuals of the
same species. The values will be inherited directly by seedlings and
will simulate fitness and spatial inheritance, as optimal values for
each patch should be selected for. Spatial inheritance is a phenomenon
in which genetically related individuals "inherit" a habitat or
continue to thrive in an area for multiple generations. This occurs when
organisms do not disperse far and become highly suited to an area \cite{Schauber2007}.

Each unoccupied, liveable patch has a chance of recruiting a seedling.
Initially, the chances of recruitment are zero for all patches. When a
mangrove reaches maturity (becomes a tree with diameter of at least 5
cm), the recruitment chance for its patch becomes 0.5. The recruitment
chance \(\Pi\) for all other patches changes during each step as governed 
by the following equation.

$$\frac{d\Pi}{\text{dt}} = \frac{- 1}{\tau}(1 - \lambda^{2}\nabla^{2})\Pi + \frac{1}{\tau}\omega$$

with $\Pi_{0} = 0$

Here, $\omega_{i}$ is a white noise term that follows Brownian motion.
It can be easily generated independently for each patch for every
step \(i\) of the simulation. For each patch,

$$\omega_{i} = \omega_{i - 1} + Normal(0,dt)$$

with $\omega_{0} = 0$


$\nabla^{2}\Pi = \Pi_{\text{xx}} + \Pi_{\text{yy}}$ can be approximated
using the finite difference method. For the boundaries, we may assume
that recruitment does not occur (this is not such a big problem since
the map used is bounded by water and an area not part of the coastal
woodlands). At step \(i\) of the simulation, let the recruitment chance
at a livable, unoccupied patch \(j,k\) be

\begin{dmath}
\Pi_{j,k}^{i} = \Pi_{j,k}^{i - 1} + dt\lbrack(\frac{- 1}{\tau})\Pi_{j,k}^{i - 1} + (\frac{\lambda^{2}}{\tau})\frac{\Pi_{j + 1,k}^{i - 1} - 2\Pi_{j,k}^{i - 1} + \Pi_{j - 1,k}^{i - 1}}{{(\Delta x)}^{2}} + (\frac{\lambda^{2}}{\tau})\frac{\Pi_{j,k + 1}^{i - 1} - 2\Pi_{j,k}^{i - 1} + \Pi_{j,k - 1}^{i - 1}}{{(\Delta y)}^{2}} + \frac{1}{\tau}\omega_{j,k}^{i}\rbrack
\end{dmath}

The simulation is initialized by creating the virtual environment and planting the initial populations of mangroves.

During each step, a Poisson-distributed random time interval \(\text{dt}\) with a mean of 1 day is generated as per \cite{SalmoJuanico2015}. Each
agent grows in diameter according to the growth equation and
proportionally to \(\text{dt}\); that is, the change in diameter during a
step in the simulation is calculated as \(\frac{\text{dD}}{\text{dt}}dt = dD\).

During the same step, the recruitment chance for each patch is
recalculated. A randomly selected mangrove may die from "natural
causes", with a probability dependent on maturity and species. A randomly
selected patch may also recruit a new seedling dependent on its
recruitment chance.

Storms also occur as a Poisson process. Each mature tree, irrespective
of species, has a probability of dying equal to \(0.7 - kn\),
where \(k\) is a parameter of the simulation and \(n\) is the number of
neighbouring trees. Smaller plants have a 0.1 chance of dying, being less vulnerable
high winds.

The model can simulate the emergence of a naturally-selected population
of mangroves over several generations, affected by both the
fragmentation of the mangrove forest (plants are not uniformly
distributed) and the properties of the plants themselves. 

The variations in the properties of each individual agent or plant, as
well as the tendency for plants that are more fit to survive longer and
have more offspring will simulate spatial selection within the
population.

The agents interact with each other by competing for resources, which
affects growth, and thus also influences mortality rates. The more
neighbours around it a plant has, the slower it should grow, and the
greater the chances of it dying during each step. However, having more
neighbours also reduces the mortality rate of the tree during storms,
since having more neighbours distributes the force and damage inflicted
by high-velocity winds.
\section{Findings from Stochastic Experiments, Briefly}
\section{Sensitivity Analysis}

A sensitivity analysis of the model was conducted. Simulations were run for 5000 steps each. The number of blocks in the block disturbance model (which represents the strength of the storms), the correlation time and length are all varied across multiple runs. The allometric parameters $\alpha$, $\beta$, and $\gamma$ are selectively varied.

The dominant population at the end of each configuration was recorded, with the results summarized in Table 4.1. N indicates the native \emph{Avicennia \& Sonneratia} population dominated, while P indicates the planted \emph{Rhizophora} population dominated. The percentage values indicate how frequently this occurred across the multiple runs.
\begin{table}
\begin{tabular}{|c|c|c|c|c|}
\hline 
Storm Strength & 0 & 1 & 2 & 3 \\ 
\hline 
Vary Alpha & N (90\%) & N (99\%) & N (90\%) & N (71\%) \\ 
\hline 
Vary Beta & N (93\%) & N (100\%) & N (97\%) & N (72\%) \\ 
\hline 
Vary Gamma & N (100\%) & N (98\%) & N (76\%) & P (79\%) \\ 
\hline 
All Constant & N (99\%) & N (98\%) & N (72\%) & P (74\%) \\ 
\hline 
\end{tabular}
\caption{Dominant populations after running for 5000 steps across multiple runs}
\end{table}

Native genera were shown to fare consistently better than \emph{Rhizophora}. 
There are two sets of runs in which the \emph{Rhizophora} dominated the
environment. From these results we may gauge the model to be sensitive to the number of blocks used in the block disturbance model, this being a primary influencing factor in determining post-disaster forest structure.

\section{Experiment}

An stochastic experiment was carried out in order to validate the model against 
field observations of post-storm disaster recovery of coastal woodlands in Pangasinan. Empirical observations indicated a recovery time of 5-7 years for the native genera (\textit{Avicennia} and \textit{Sonneratia}). The non-native genera (\textit{Rhizophora}) only recovered up to 40\% around 7 years after the storm. Recovery is measured as the percentage of total forest cover regained post-storm.

In the simulation, total forest cover is calculated as the sum of crown areas for all mangroves in 
a population. Crown area (or the area covered by the tree's foliage) can be estimated using the
plant's diameter at breast height \cite{berger}, where: $r_{crown} = 11.1D^{0.645}$. The crown area is calculated using the crown radius (in square centimeters) with the assumption that the tree crown is a circular region.

\textbf{INSERT TABLES HERE}

Overall, however, the model is an example of how the stochastic processes discussed in CS 175 can be applied to complex systems--especially ones where a definite and complete idea of how the system behaves is difficult to come by. In complex systems such as mangrove forests, the conversion rates (such as mortality and birth) are always changing and difficult to predict without a vast amount of empirical data, which is admittedly hard to collect. Agent-based models allow for a robust simulation of such systems. Stochasticity allow these models to further take into consideration the many unaccounted-for variables that in reality may act upon the system being modelled.

\end{multicols}

\label{sect:bib}
\bibliographystyle{plain}
\bibliography{mybib}

\end{document}