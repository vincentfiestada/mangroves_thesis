\documentclass[a4paper]{scl-article}
\usepackage{amsmath,amssymb,graphicx,array,multirow,float,multicol,breqn}
\usepackage{xwatermark,tikz}

%\newsavebox\mybox
%\savebox\mybox{\tikz[color=gray,opacity=0.2]\node{DRAFT};}
%\newwatermark[
%  allpages,
%  angle=45,
%  scale=6,
%  xpos=-20,
%  ypos=15
%]{\usebox\mybox}

\begin{document}

\title{A Stochastic Agent-based Model of Mangrove Forest Regrowth: A Summarized Report for CS 175}

\author{Vincent Paul Fiestada, Andrew Vince Lorbis\\Adviser: Vena Pearl Bongolan, Ph.D.}

\maketitle

\begin{multicols}{2}
Mangroves are woody plants that live in the coastal intertidal zone.
They are used in flavouring agents, textiles, housing, and more. Their root systems
also serve as a habitat for many species of fish. 
The mangroves themselves are a key part of mangal ecosystems. Some flora and fauna associated
with mangroves are algae, sea grasses, saltmarsh plants, sponges, nematodes, prawns, shrimp,
crabs, dragonflies, bees, ants, shipworms, and fish. 
There is high genetic diversity among mangroves, owing to the necessity of adapting to their habitats \cite{biologyOfMangroves}. In recent years, mangrove populations have significantly
declined--especially in developing countries, where about 90\% of them are found \cite{kavanagh2007}.

There are various factors that affect the growth of mangroves, including tidal
inundation, salinity, competition between other mangroves, and storms. 

There is a need for a stochastic model that can predict the regenerative
behavior of multi-species mangrove forests in a fragmented
habitat. Such a model can help inform the decision-making process of environmental
conservation efforts. The authors improve upon work done by Salmo \& Juanico \cite{SalmoJuanico2015} and an agent-based implementation by Ang \& Mariano \cite{mangrovesAngMariano} in order to create such a model. This is a shorter and summarized
version for CS 175.
\section{Model Overview, Design concepts, and Details}
To create the model, the authors draw heavily on an existing mathematical model
of individual mangrove growth by Salmo and Juanico \cite{SalmoJuanico2015} and make improvements to an agent-based multi-specific implementation by Ang and Mariano \cite{mangrovesAngMariano}. Stochasticity has been added by generating spatio-temporal coloured noise for seedling
dispersal; spatial inheritance is simulated by a variance in allometric attributes of individual mangrove agents. The model is implemented in NetLogo and allows for a high level of customization.

Further improvements made to the model are differentiation among the recruitment of
viviparous species; increased resistance to storm damage among
clustered trees and the block disturbance model applied to storm damage.

The model aims to be a generalized simulator for the growth or regeneration of a
multi-specific mangrove forest in a fragmented habitat. In particular,
the \emph{Avicennia, Sonneratia,} and \emph{Rhizophora} species are
investigated in a virtual analogue of the Bangrin Marine Protected Area in
Bani, Pangasinan.

\section{Agents}

The main agent being modelled is the mangrove plant. These mangrove agents
have a single state variable that determines their growth and maturity:
the diameter at breast height of the mangrove, which we will refer to
as\(D\). Empirical data can be used to allometrically relate D to other
factors such as tree height and crown radius \cite{Hiebeler2000}.

\subsection{Agent Classification}

Agents can be classified in two ways: by species and by maturity. As the
model deals with multi-specific forests, the behaviour of mangrove
agents is modified by species-specific parameters. Mangroves are also
divided into maturity levels. In Ang and Mariano \cite{mangrovesAngMariano}, there are
three maturity levels (dependent on \(D\)) ranging from seedling to tree,
with different mortality rates tied to each maturity level per species.

In this model, the chances that a tree will die from natural causes is 
scales along with its \(D\). Each maturity level for each species has an assigned value
to represent the chance of dying from natural causes.
A Lagrange interpolated mortality rate from those values is obtained, so that as a
plant grows larger (i.e. $D$ increases) its chance of dying also steadily decreases.

\section{Spatial Units}

The model simulates mangrove agents in a closed patch grid consisting of
some land area and sea area, separated by an arbitrarily shaped coast. Some patches
are considered as non-liveable (e.g. patches that represent the sea or land used by humans).

\subsection{Coastal Distance}

Salinity and tidal inundation responses are approximately dependent on
distance from the coast. The farther from the coast a patch is, the less
inundated and the less saline it should be due to having less exposure
to seawater.

Each patch thus has a specific inundation and salinity effect to the
growth of the mangrove plant on it.

\section{Variables}

The model's determination of individual mangrove growth follows the
differential equation proposed by Salmo and Juanico \cite{SalmoJuanico2015}. The
deterministic change in $D$ with respect to time is as follows:


\begin{dmath}
\frac{\text{dD}}{\text{dt}} = (\frac{\Omega}{2 + \alpha})D^{\beta - \alpha - 1}\lbrack 1 - \frac{1}{\gamma}{(\frac{D}{D_{\max}})}^{1 + \alpha}\rbrack \times \sigma(x,y) \times \eta(x,y) \times K(x,y)
\end{dmath}


\(\alpha\)- allometric constant relating $D$ to tree height

\(\beta\)- allometric constant relating $D$ to crown radius

\(\Omega\)- conversion factor that harmonizes both sides of the equation

\(\gamma\)- species-specific constant related to maximum height

\(D_{\max}\)- species-specific constant indicating maximum attainable $D$

\(\sigma(x,y)\)- response to salinity

\(\eta(x,y)\)- response to inundation

\(K(x,y)\)- response to competition from nearby mangroves

Salmo and Juanico also provide formulas for calculating the responses to
salinity, tidal inundation and competition from patch values and
neighbourhood characteristics. The authors have left these unchanged.

\begin{dmath}
\sigma(x,y) = \left[ 1 + \exp{\left( \frac{S(x,y) - \tilde{S}}{\Delta S} \right) } \right] ^{-1}
\end{dmath}
\paragraph{Salinity response.} Here, $\tilde{S}$ is the maximum salinity level that above which the plant will no longer grow and $\Delta S$ is the species-specific tolerance to salinity

\begin{dmath}
\eta(x,y) = 1 - I(x,y)
\end{dmath}
\paragraph{Inundation response.} Here, $I(x,y)$ is the inundation value at patch $(x,y)$; mangroves are expected to grow faster if experiencing less tidal inundation (assuming it is still sufficiently hydrated)

\begin{dmath}
K(x,y) = 1 - 2\sum_{N}{f(x,y)}
\end{dmath}
\paragraph{Competition response.} In which $f(x,y)$ is the competition field around the focal mangrove at patch $(x,y)$. This is calculated as $\exp{\left[ -c(r - \frac{D}{2}) \right]}$ for neighbours within a radial distance ranging from $\frac{D}{2}$ up to the crown radius. $N$ is the local neighbourhood of the plant, or the set of mangroves within the said radius.

\paragraph{Spatial Inheritance.} The allometric parameters \(\alpha\), \(\beta\) and \(\gamma\) will be species-specific but will also vary randomly among individuals of the
same species. The values are inherited by seedlings in order to 
simulate fitness and spatial inheritance, as optimal values become selected over time. Spatial inheritance is a phenomenon in which genetically related individuals "inherit" a habitat or
continue to thrive in an area for multiple generations. This occurs when
organisms do not disperse far and become highly suited to an area \cite{Schauber2007}.

\subsection{Recruitment}

Each unoccupied, liveable patch has a chance of recruiting a seedling.
Initially, the chances of recruitment is zero for all patches. When a
mangrove reaches maturity (becomes a tree with diameter of at least 5
cm), the recruitment chance for its patch becomes 0.5. The recruitment
chance \(\Pi\) for all other patches changes as governed by the following
equation.

$$\frac{d\Pi}{\text{dt}} = \frac{-1}{\tau}(1 - \lambda^{2}\nabla^{2})\Pi + \frac{1}{\tau}\omega$$

with $\Pi_{0} = 0$

Here, $\omega_{i}$ is a white noise term that follows Brownian motion.
It can be easily generated independently for each patch for every
step\(i\)of the simulation. For each patch,

$$\omega_{i} = \omega_{i - 1} + Normal(0,dt)$$

with $\omega_{0} = 0$


$nabla^{2}\Pi = \Pi_{\text{xx}} + \Pi_{\text{yy}}$ can be approximated
using the finite difference method. For the boundaries, we may assume
that recruitment does not occur (this is not such a big problem since
the map used is bounded by water and land area not part of the coastal
woodlands). At step \(i\) of the simulation, let the recruitment chance
at a livable, unoccupied patch \(j,k\) be

\begin{dmath}
\Pi_{j,k}^{i} = \Pi_{j,k}^{i - 1} + dt\lbrack(\frac{- 1}{\tau})\Pi_{j,k}^{i - 1} + (\frac{\lambda^{2}}{\tau})\frac{\Pi_{j + 1,k}^{i - 1} - 2\Pi_{j,k}^{i - 1} + \Pi_{j - 1,k}^{i - 1}}{{(\Delta x)}^{2}} + (\frac{\lambda^{2}}{\tau})\frac{\Pi_{j,k + 1}^{i - 1} - 2\Pi_{j,k}^{i - 1} + \Pi_{j,k - 1}^{i - 1}}{{(\Delta y)}^{2}} + \frac{1}{\tau}\omega_{j,k}^{i}\rbrack
\end{dmath}

The values for the correlation time and length are varied throughout the
experiments.

\section{Process Overview}

This section provides an overview of how the model 
The simulation is initialized by creating the virtual environment (with
salinity and inundation values calculated for each patch using their
distance from the sea) and planting the initial populations or
mangroves.

During each step, a Poisson-distributed random time interval \(\text{dt}\) 
with a mean of 1 day is generated as per \cite{SalmoJuanico2015}. Each
agent/mangrove grows in diameter according to the growth equation and
proportionally to \(\text{dt}\); that is, the change in diameter during a
step in the simulation is calculated as \(\frac{\text{dD}}{\text{dt}}dt = dD\).

During the same step, the recruitment chance for each patch is
recalculated. A randomly selected mangrove may die from natural
causes, with a probability dependent on its maturity, as previously described. A randomly
selected patch may also recruit a new seedling dependent on its
recruitment chance.

Storms also occur as a Poisson process. Each mature tree, irrespective
of species, has a probability of dying equal to \(0.7 - kn\),
where \(k\) is a parameter of the simulation and \(n\) is the number of
neighbouring trees within the focal tree's field of neighbourhood. Smaller
plants have a 0.1 chance of dying.

\section{Design Concepts}

Storms and death are both modelled as natural random processes with
varying rates depending on species and maturity. Seedling dispersal is
modelled as diffusion of propagules using spatio-temporal coloured noise.
Block disturbance is used to model the effects of storms on the mangrove
populations. Spatial inheritance, or the selection of the fittest organisms living
in a certain area, is simulated by the variance of allometric parameters
between individuals.

\subsection{Emergence and Adaptation}
The model can simulate the emergence of a naturally-selected population
of mangroves over several generations, affected by both the
fragmentation of the mangrove forest (plants are not uniformly
distributed) and the properties of the plants themselves. It will thus
be able to take into account the effects of having multiple species and
an arbitrary landscape.

The variations in the properties of each individual agent or plant, as
well as the tendency for plants that are more fit to survive longer and
have more offspring will simulate spatial selection within the
population.

\subsection{Interaction}
Plant agents interact with each other by competing for resources, which
affects growth, and thus also influences mortality rates. The more
neighbours around it a plant has, the slower it should grow, and thus the
greater the chances of it dying during each step. However, having more
neighbours also reduces the mortality rate of the tree during storms,
since having more neighbours distributes the force and damage inflicted
by the storm.

\section{Findings from Stochastic Experiments, Briefly}
Since this is a summarized report that is more concerned with the inner workings
of the model and the applications of stochastic simulations, there will be no in-depth
discussion of the results of experiments. For more information, please read the full version. 

A sensitivity analysis of the model was conducted. Simulations were run for 5000 steps, varying the number of blocks in the block disturbance model and the allometric
parameters $\alpha$, $\beta$, and $\gamma$ in a simulation of 100 runs. It should be noted that when $\alpha$ or $\beta$ were varied, the native genera dominated the environment. When $\gamma$ was varied, the non-native (planted) genera dominated.

The dominant population at the end of each configuration was also recorded. \emph{Avicennia \& Sonneratia}, the native genera, dominated in most configurations, as shown the table below. N indicates that the native genera dominated and P indicates that the planted mangroves \emph{Rhizophora} dominated. The number in the parentheses indicate how many times the species dominated.

\resizebox{0.45 \textwidth}{!}{%
\begin{tabular}{|c|c|c|c|c|}
\hline 
Storm Strength & 0 & 1 & 2 & 3 \\ 
\hline 
Vary Alpha & N (90\%) & N (99\%) & N (90\%) & N (71\%) \\ 
\hline 
Vary Beta & N (93\%) & N (100\%) & N (97\%) & N (72\%) \\ 
\hline 
Vary Gamma & N (100\%) & N (98\%) & N (76\%) & P (79\%) \\ 
\hline 
All Constant & N (99\%) & N (98\%) & N (72\%) & P (74\%) \\ 
\hline 
\end{tabular}}

Note here that the planted species fare better when storms are \emph{stronger},
since the native species have more established trees and thus are more vulnerable to
storms. After such a storm has devastated the native population, the non-native species
has a chance to thrive.

Native genera were shown to fare consistently better
than \emph{Rhizophora} when all of these mangrove populations were mixed
in the planted site. They displayed stable or slowly increasing
population numbers, the random occurrence of powerful typhoons
notwithstanding. 

In terms of regeneration, when counting the number of mature trees before each storm
and comparing the successive population growth, the native mangroves take about 4-5 years
to recover and regain the same number of mature trees. Planted mangroves take about 2-4 years to regain the number of trees after storms.

It should be noted that as storms become
stronger, the recovery times tend to go down, because the number of trees decrease and 
successive regrowth requires less and less trees to regain those numbers. The faster recovery rate of the planted mangroves can
be attributed to the same phenomenon. Their small population numbers mean that they need only a small amount of time before reaching the pre-disaster tree population again. More experiments will be done, counting seedlings and saplings, and making more mature trees count for more, in order to account for this. This is similar to the method of looking at foliage cover (more mature trees contribute more foliage) and would allow us to take into consideration the amount of biomass that needs to be regrown before pronouncing a population as "recovered".

Overall, however, the model is an example of how the stochastic processes discussed in CS 175 can be applied to complex systems--especially ones where a definite and complete idea of how the system behaves is difficult to come by. In complex systems such as mangrove forests, the conversion rates (such as mortality and birth) are always changing and difficult to predict without a vast amount of empirical data, which is admittedly hard to collect. Agent-based models allow for a robust simulation of such systems. Stochasticity allow these models to further take into consideration the many unaccounted-for variables that in reality may act upon the system being modelled.

\end{multicols}

\label{sect:bib}
\bibliographystyle{plain}
\bibliography{mybib}

\end{document}