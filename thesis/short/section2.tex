In the Block Disturbance model by Hiebeler \cite{HiebelerFragmentedHabitat2011}, 
agents are placed in an environment consisting of a lattice of tiles that individuals can
occupy, and periodic disturbances which cause their death. This model
can be used for storms (which is a Poisson process) that occur randomly
to damage the mangrove populations. Every time a storm occurs, an area
in the mangrove forest is more heavily damaged than other areas. This
randomness could be likened to the different paths a storm or typhoon
can take while passing through the coastal woodlands.