%  SCL LaTeX Poster Class.
%  20170203 v1.1.0
%  Copyright 2015-2017 by John Justine S. Villar <john_justine.villar@up.edu.ph>
%
%  This is free software: you can redistribute it and/or modify
%  it under the terms of the GNU General Public License as published by
%  the Free Software Foundation, either version 3 of the License, or
%  (at your option) any later version.
%
%  This is distributed in the hope that it will be useful,
%  but WITHOUT ANY WARRANTY; without even the implied warranty of
%  MERCHANTABILITY or FITNESS FOR A PARTICULAR PURPOSE.  See the
%  GNU General Public License for more details.
%
%  You can find the GNU General Public License at <http://www.gnu.org/licenses/>.


\documentclass[a0paper,portrait]{scl-poster}
%COLOR SCHEMES----------------------------------
% SCLcolors: SCL default color scheme (SCLdarkblue)
% SCLcolors1: SCL green scheme
% SCLcolors2: SCL terra cotta scheme
% SCLcolors3: SCL purple scheme
% SCLcolors4: SCL light blue scheme
% SCLcolors5: SCL gray scheme
% UPcolors:  UP color scheme (UP maroon, UP green, as per UP brand book)
%-----------------------------------------------
\usepackage[SCLcolors]{SCLpostersetup}


\begin{document}
\begin{poster}
%****************************************************************
%SETTINGS (Note: Please do not edit linebreaks and spaces within this section!)
%
%GENERAL POSTER SETTINGS
{%
  columns=3,
  headerheight=0.12\textheight, %top header, default = 0.1\textheight
  linewidth=1pt   %for posterbox 
}%
 
%TITLE HEADER SETTINGS
{\color{white}\bf\huge %xxxxxxxxxxxxxxx
%TITLE------------------------------------------
scl-poster: SCL \LaTeX\ Poster Theme
%-----------------------------------------------
\\[0.005em]}   %xxxxxxxxxxxxxxxxxxxxxxxxx
{\color{white}{%xxxxxxxxxxxxxxxxxxxxxxxxx
%AUTHOR-----------------------------------------
John Justine S. Villar$^1$, Adrian Roy Valdez$^1$ and Collaborator's Name$^2$
%----------------------------------------------------
}\\[0.2em]\small %xxxxxxxxxxxxxxxxxxxxxx
%INSTITUTE--------------------------------------
$^1$Scientific Computing Laboratory, Department of Computer Science, University of the Philippines\\
$^2$Collaborator's Institution
%-----------------------------------------------
\\\tt %xxxxxxxxxxxxxxxxxxxxxxxxxxxxxxxxx
%EMAIL-------------------------------------------
john\_justine.villar@upd.edu.ph, alvaldez@dcs.upd.edu.ph, collaborator@otheruni.edu
%-----------------------------------------------
}
%LOGO (the logo on the right)
{
\dcsscllogowhite
%\dcsscllogo
}
%****************************************************************



%POSTER BODY------------------------------------
\begin{posterbox}[name=intro,column=0,row=0]{Introduction}
\begin{itemize}
  \item Posters are often used at conferences for presenting exciting new research results.
  \item So far no (un)official poster theme is available to the members of the Scientific Computing Laboratory (SCL).
  \item The present theme is an attempt to change this.
  \item The \alert{SCLposter} theme is a minor modification of the \alert{baposter} poster template which you can download at \url{http://www.brian-amberg.de/uni/poster/}.
\end{itemize}
\end{posterbox}

\begin{posterbox}[name=usage,column=0,below=intro]{Usage}
\begin{itemize}
  \item To use the \alert{SCLposter} theme, place the {\tt sclpostermain.tex} file in your preferred folder and modify the file to your needs.
  \item You can read more about how you can modify the theme in the documentation for the \alert{baposter} template which you can find here \url{http://www.brian-amberg.de/uni/poster/}.
\end{itemize}
\end{posterbox}

\begin{posterbox}[name=lists,column=0,below=usage]{Lists}
Itemize
\begin{itemize}
  \item item 1
    \begin{itemize}
      \item subitem 1
        \begin{itemize}
          \item subsubitem 1
            \begin{itemize}
              \item subsubsubitem 1
              \item subsubsubitem 2
            \end{itemize}
          \item subsubitem 2
        \end{itemize}
      \item subitem 2
    \end{itemize}
  \item item 2
\end{itemize}
Enumerate
\begin{enumerate}
  \item item 1
    \begin{enumerate}
      \item subitem 1
        \begin{enumerate}
          \item subsubitem 1
            \begin{enumerate}
              \item subsubsubitem 1
              \item subsubsubitem 2
            \end{enumerate}
          \item subsubitem 2
        \end{enumerate}
      \item subitem 2
    \end{enumerate}
  \item item 2
\end{enumerate}
Description
\begin{description}
  \item[desc 1] item 1
    \begin{description}
      \item[desc 1] subitem 1
        \begin{description}
          \item[desc 1] subsubitem 1
            \begin{description}
              \item[desc 1] subsubsubitem 1
              \item[desc 2] subsubsubitem 2
            \end{description}
          \item[desc 2] subsubitem 2
        \end{description}
      \item[desc 2] subitem 2
    \end{description}
  \item[desc 2] item 2
\end{description}
\end{posterbox}

\begin{posterbox}[name=equation,column=0,below=lists,above=bottom]{Equations}
Here is an example of an equation
\begin{equation}
  f_X(x|\mu,\sigma^2) = \frac{1}{\sqrt{2\pi\sigma^2}}\exp\left\{\frac{1}{2\sigma^2}(x-\mu)^2\right\}
\end{equation}
\end{posterbox}

\begin{posterbox}[name=install,span=2,column=1,row=0]{Installation}
You can either make a local or a global installation of the \alert{SCLposter} template \cite{sclposter}.
\begin{description}
  \item[Local:] Place the {\tt SCLposter.cls} file in the same folder as the poster file {\tt SCLposter.tex}
  \item[Global:] Place the {\tt SCLposter.cls} file in your local latex-directory tree. This is by default {\tt <somewhere>/textmf/tex/latex/SCLposter} where {\tt <somewhere>} is
  \begin{description}
    \item[GNU/Linux:] {\tt/home/<username>}
    \item[Windows XP:] {\tt c:\textbackslash Document and Settings\textbackslash<username>}
    \item[Windows Vista+:] {\tt c:\textbackslash Users\textbackslash<username>}
    \item[Mac OSX:] {\tt/home/<username>/Library}
  \end{description}
  On GNU/Linux and Windows, you have to update the filename database after placing {\tt SCLposter.cls} in the correct folder. This is done by
  \begin{description}
    \item[GNU/Linux:] {\tt \$ texhash \textasciitilde /texmf}
    \item[Windows with MiKTeX:] Open the MiKTeX Settings dialog and click 'Refresh FNDB'.
    \item[Windows with TeX Live:] Open the TeX Live Manager dialog and select 'Update filename database' under 'Actions'.
  \end{description}
\end{description}
\end{posterbox}

\begin{posterbox}[name=figures,column=1,below=install,above=bottom]{Figures and Tables}
You cannot use floats in the \alert{SCLposter} template. However, you can use figure captions by using {\tt \textbackslash captionof} instead of {\tt \textbackslash caption}. This is demonstrated in Fig.~\ref{fig:figlabel}. Moreover, you can also use {\tt \textbackslash label} and {\tt \textbackslash ref} to make references to your figures and/or tables.
\begin{center}
  \includegraphics[width=0.5\textwidth]{graphics/uplogo}
  \captionof{figure}{Here is a figure caption}
  \label{fig:figlabel}
\end{center}
You can of course change the background color through the {\tt boxColorOne} option. Alternatively, you can make the background transparent. In Matlab, the following example demonstrates how this is done\par
{\tt
f1 = figure(1);\\
set(f1,'Color','none');
}\par
You can also use {\tt pgfplots} for plotting your Matlab data. This is not that hard and the resulting plots are much nicer than Matlab plots, so I will strongly recommend that you have a look at {\tt pgfplots} at \url{http://sourceforge.net/projects/pgfplots/}.
\begin{center}
  \begin{tabular}{c c c}
    \toprule
    header 1 & header 2 & header 3\\
    \midrule
    data (1,1) & data (1,2) & data (1,3)\\
    data (2,1) & data (2,2) & data (2,3)\\
    data (3,1) & data (3,2) & data (3,3)\\
    \bottomrule
  \end{tabular}
  \captionof{table}{A very simple table with booktabs}
  \label{tab:tablabel}
\end{center}
\end{posterbox}

\begin{posterbox}[name=feedback,column=2,below=install]{Feedback}
  \begin{itemize}
    \item The \alert{SCLposter} theme is based on and has been tested with \alert{baposter} v.2.0, and it can be downloaded from the SCL website.
    \item If you find a bug in the SCL poster theme (and not in the \alert{baposter} template), please do not hesitate to contact me. There is a FAQ at the \alert{baposter} website, if you should have any problems with it.
  \end{itemize}
\end{posterbox}

\begin{posterbox}[name=refs,column=2,below=feedback,above=bottom]{References}
% In the last box, you will usually have a list of references
% The bibliography automatically adds the title "References", but
% this have been removed in the preamble

\vspace{0.7em}

% use either ......
%{\scriptsize\singlespacing %
%\begin{thebibliography}{1}% Simple bibliography with widest label of 1
%\itemsep=-0.01em% Save space between the separation
%\setlength{\baselineskip}{0.4em}% Save space with longer lines
%\bibitem{baposter} Brian Amberg: \emph{LaTeX Poster Template}, \url{http://www.brian-amberg.de/uni/poster/} 
%\bibitem{pgfplots} Christian Feuersänger: \emph{PGFPlots - A LaTeX Package to create normal/logarithmic plots in two and three dimensions}, \url{http://pgfplots.sourceforge.net/} 
%\bibitem{sclposter} John Justine Villar: \emph{SCL LaTeX Templates}, \url{http://scl.dcs.upd.edu.ph/internal}
%\end{thebibliography}}


% or uncomment below, if you are usin a .bib file:
{\scriptsize\singlespacing
\bibliographystyle{plain}
\bibliography{references}}
\end{posterbox}


\end{poster}
\end{document}
