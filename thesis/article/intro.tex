According to the National Oceanic and Atmospheric Administration,
Mangroves are trees and shrubs that live in the coastal intertidal zone.
The root system of mangrove forests serve as a habitat to various
aquatic organisms. Mangroves are used in flavoring agents, textiles,
housing, and more.

Mangrove seedlings can float and take root once it finds ground, and can
disperse and grow far away from parent tree. Some species have seeds
that develop while still attached to the parent plant, and break off
afterwards. Some mangroves have been known to live for more than 100
years. However, most trees will never reach this age because of various
factors affecting mangrove growth and mortality. These include tidal
inundation, salinity, competition between other mangroves, and storms.

There is a need for a stochastic model that can predict the regenerative
behavior of single and multi-species mangrove forests in a fragmented
habitat. Such a model can help inform the decision-making process of environmental
conservation efforts. In this paper, we improve upon a model for individual growth
developed by Salmo \& Juanico \cite{SalmoJuanico2015} and an agent-based 
implementation by Ang \& Mariano \cite{mangrovesAngMariano}.