Mangrove forests along tropical coastlines frequently suffer severe
damage due to storms. The mangrove trees are extremely sensitive to
environmental stressors such as water salinity and tidal inundation. We
propose an agent-based model for the prediction of the regenerative
behavior of mangrove stands consisting of the native species and the
planted or non-native species in a fragmented habitat, with the use of
spatiotemporal colored noise for stochastic seedling dispersal. This
study aims to model accurately the growth of mangrove populations while
experiencing said factors and subject to storm damage. It uses Salmo and
Juanico's model for individual mangrove growth, and uses spatiotemporal
colored noise for the dispersal of seedlings. Stochastic experiments were carried out in
a shoreline habitat with an existing native population (\emph{Avicennia}
and \emph{Sonneratia}) of varying ages and a larger population of
planted, non-native seedlings (\emph{Rhizophora}).

A sensitivity analysis shows that in a run lasting 5000 days, there are
two outcomes depending on what variable is varying in the growth
equation. When either of the allometric parameters\(\alpha\)or
\(\beta\)(related to tree height and crown radius) are varying, the
native species dominates the environment, with both species taking
roughly 3.5 years for trees to recover from storms. When everything is
constant, or \(\lambda\)(allometric parameter related to maximum height)
is varying, the non-native species dominates the environment.

In a Monte Carlo experiment where the allometric parameters, correlation
time and diffusion rate (both used in the spatiotemporal colored noise
associated with seedling dispersal chance) are varying, the native
species dominated the environment nearly 77\% of the time with an
average recovery time of 3.98 years.