A sensitivity analysis of the model was conducted. Simulations were run for 5000 steps each.
The number of blocks in the block disturbance model, the correlation time and length, and the allometric parameters $\alpha$, $\beta$, and $\gamma$ are all varied.

The dominant population at the end of each configuration was recorded, with the results summarized in the table below. N indicates the native \emph{Avicennia \& Sonneratia} population dominated, while P indicates the planted \emph{Rhizophora} population dominated. The number
in the parentheses indicate how frequently this occurred across the multiple runs.

\resizebox{0.45 \textwidth}{!}{%
\begin{tabular}{|c|c|c|c|c|}
\hline 
Storm Strength & 0 & 1 & 2 & 3 \\ 
\hline 
Vary Alpha & N (90\%) & N (99\%) & N (90\%) & N (71\%) \\ 
\hline 
Vary Beta & N (93\%) & N (100\%) & N (97\%) & N (72\%) \\ 
\hline 
Vary Gamma & N (100\%) & N (98\%) & N (76\%) & P (79\%) \\ 
\hline 
All Constant & N (99\%) & N (98\%) & N (72\%) & P (74\%) \\ 
\hline 
\end{tabular}}

There are two sets of runs in which the \emph{Rhizophora} dominated the
environment, both occurring when three (3) blocks were used for 
being disturbed. Native genera were shown to fare consistently better
than \emph{Rhizophora} when all of these mangrove populations were mixed
in the planted site. They displayed stable or slowly increasing
population numbers, the random occurrence of powerful typhoons
notwithstanding. The faster recovery rate of the planted mangroves can
be attributed to its small population. Since they initially had a small
population, it stands to reason that they need only a small amount of
time before reaching the pre-disaster population again.

\subsection{Experiment}

An stochastic experiment was carried out in order to validate the model against 
field observations of post-storm disaster recovery of coastal woodlands in Pangasinan. Empirical observations indicated a recovery time of 5-7 years for
the native genera (\textit{Avicennia} and \textit{Sonneratia}). The non-native genera (\textit{Rhizophora}) only recovered up to 40\% around 7 years after the storm. Recovery is measured as the percentage of total forest cover regained post-storm.

In the simulation, total forest cover is calculated as the sum of crown areas for all mangroves in 
a population. Crown area (or the area covered by the tree's foliage) can be estimated using the
plant's diameter at breast height \cite{Hiebeler2000}, where: $r_{crown} = 11.1D^{0.645}$. The crown area is calculated using the crown radius (in square centimeters).

\section{Conclusions and
Recommendations}\label{conclusions-and-recommendations}

This study shows that planting the wrong species of mangroves in an area not
suited for them would not yield good results from the point of view of
population density and thus, biomass. It is therefore important to
choose the right type of species to plant when planning mangrove
rehabilitation projects. It could also be seen that the block
disturbance plays a huge factor on the population of the mangroves. The
native mangroves, \emph{Avicennia} and \emph{Sonneratia}, will continue
to thrive in the environment, but the experiments show that a storm
strong enough to heavily damage the native genera will cause the
\emph{Rhizophora mucronata} mangroves to thrive and take over the
habitat. They are especially vulnerable to storms since they initially
have a larger number of trees that can be affected by typhoons.\\
The researchers applied a one-to-one correspondence between the matrix
plot unit and the agent. This means that only one tree can grow on one
patch at a given time. Letting multiple trees thrive on one patch can be
added as an improvement to further study the mangrove's propagation and
its growth with response to competition. The researchers further
recommend applying the model to other types of virtual environments and
species if empirical data is available.