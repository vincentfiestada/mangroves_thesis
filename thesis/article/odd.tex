To create the model, we draw heavily on an existing mathematical model
of mangrove growth by Salmo and Juanico \cite{SalmoJuanico2015} and make improvements to
an agent-based implementation by Ang and Mariano \cite{mangrovesAngMariano}. Stochasticity
has been added by generating spatiotemporal colored noise for seedling
dispersal; spatial inheritance is simulated by a variance in allometric
attributes of individual mangrove agents. The model is implemented in
NetLogo and allows for a high level of customization.

Other improvements include differentiation among the recruitment of
viviparous species and increased resistance to storm damage among
clustered trees.

The model aims to be a generalized model of growth or regeneration of a
hetero-specific mangrove forest in a fragmented habitat. In particular,
the \emph{Avicennia, Sonneratia,} and \emph{Rhizophora} species are
investigated in a virtual analog of the Bangrin Marine Protected Area in
Bani, Pangasinan.

\subsection{Agents}

The main agent being modeled is the mangrove plant. These mangrove agents
have a single state variable that determines their growth and maturity:
the diameter at breast height of the mangrove, which we will refer to
as\(D\). Empirical data can be used to allometrically relate D to other
factors such as tree height and weight \cite{Hiebeler2000}.

\subsubsection{Agent Classification}

Agents can be classified in two ways: by species and by maturity. As the
model deals with hetero-specific forests, the behavior of mangrove
agents is modified by species-specific parameters. Mangroves are also
divided into maturity levels. In Ang and Mariano {[}7{]}, there were
three maturity levels (dependent on\(D\)) ranging from seedling to tree,
with different mortality rates tied to each maturity level per species.

Instead, we forego the species-specific mortality rates. The chances
that a tree will die from natural causes is now tied solely to its\(D\).
In \cite{mangrovesAngMariano}, \emph{Sonneratia} and \emph{Avicennia} trees (with
\(D \geq 5\)cm) had a 0.166 chance of dying from natural causes; for
saplings (\(D \in \lbrack 2.5,5)\)), it was 0.2, and for seedlings
(\(D \in \lbrack 0.5,2.5)\)), it was 0.4. \emph{Rhizophora} had slightly
higher mortality rates for each maturity level. The same general trend
of higher mortality for less mature plants is followed all mangroves. In
the present model, we obtain a Lagrange interpolated mortality rate from
these values, so that a plant's chances of dying from natural causes
gradually but steadily declines as its\(D\)increases.

\subsection{Spatial Units}

The model simulates mangrove agents in a closed patch grid consisting of
some land area and sea area, separated by an arbitrarily shaped coast.

\subsubsection{Coastal Distance}

Salinity and tidal inundation responses are approximately dependent on
distance from the coast. The farther from the coast a patch is, the less
inundated and the less saline it should be due to having less exposure
to seawater.

Each patch thus has a specific inundation and salinity effect to the
growth of the mangrove plant on it.

\subsection{Variables}

The model's determination of individual mangrove growth follows the
differential equation proposed by Salmo and Juanico {[}1{]}. The
deterministic change in D with respect to time is as follows:


\begin{dmath}
\frac{\text{dD}}{\text{dt}} = (\frac{\Omega}{2 + \alpha})D^{\beta - \alpha - 1}\lbrack 1 - \frac{1}{\gamma}{(\frac{D}{D_{\max}})}^{1 + \alpha}\rbrack \times \sigma(x,y) \times \eta(x,y) \times K(x,y)
\end{dmath}


\(\alpha\)- allometric constant relating D to tree height

\(\beta\)- allometric constant relating D to crown radius

\(\Omega\)- conversion factor that harmonizes both sides of the equation

\(\gamma\)- species-specific constant related to maximum height

\(D_{\max}\)- species-specific constant indicating maximum attainable D

\(\sigma(x,y)\)- response to salinity

\(\eta(x,y)\)- response to inundation

\(K(x,y)\)- response to competition from nearby mangroves

Salmo and Juanico also provide formulas for calculating the responses to
salinity, tidal inundation and competition from patch values and
neighborhood characteristics. The authors have left these unchanged.

The allometric parameters\(\alpha\), \(\beta\)and \(\gamma\)will be
species-specific but will also vary randomly among individuals of the
same species. The values will be inherited directly by seedlings and
will simulate fitness and spatial inheritance, as optimal values for
each patch should be selected for. Spatial inheritance is a phenomenon
in which genetically related individuals ``inherit'' a habitat or
continue to thrive in an area for multiple generations. This occurs when
organisms do not disperse far and become highly suited to an area \cite{Schauber2007}.

\subsubsection{Recruitment}

Each unoccupied, livable patch has a chance of recruiting a seedling.
Initially, the chances of recruitment is zero for all patches. When a
mangrove reaches maturity (becomes a tree with diameter of at least 5
cm), the recruitment chance for its patch becomes 0.5. The recruitment
chance\(\Pi\)for all other patches changes as governed by the following
equation.

$$\frac{d\Pi}{\text{dt}} = \frac{- 1}{\tau}(1 - \lambda^{2}\nabla^{2})\Pi + \frac{1}{\tau}\omega$$

with $\Pi_{0} = 0$

Here, $\omega_{i}$ is a white noise term that follows Brownian motion.
It can be easily generated independently for each patch for every
step\(i\)of the simulation. For each patch,

$$\omega_{i} = \omega_{i - 1} + Normal(0,dt)$$

with $\omega_{0} = 0$


$nabla^{2}\Pi = \Pi_{\text{xx}} + \Pi_{\text{yy}}$ can be approximated
using the finite difference method. For the boundaries, we may assume
that recruitment does not occur (this is not such a big problem since
the map used is bounded by water and area not part of the coastal
woodlands). At step \(i\)of the simulation, let the recruitment chance
at a livable, unoccupied patch \(j,k\) be

\begin{dmath}
\Pi_{j,k}^{i} = \Pi_{j,k}^{i - 1} + dt\lbrack(\frac{- 1}{\tau})\Pi_{j,k}^{i - 1} + (\frac{\lambda^{2}}{\tau})\frac{\Pi_{j + 1,k}^{i - 1} - 2\Pi_{j,k}^{i - 1} + \Pi_{j - 1,k}^{i - 1}}{{(\Delta x)}^{2}} + (\frac{\lambda^{2}}{\tau})\frac{\Pi_{j,k + 1}^{i - 1} - 2\Pi_{j,k}^{i - 1} + \Pi_{j,k - 1}^{i - 1}}{{(\Delta y)}^{2}} + \frac{1}{\tau}\omega_{j,k}^{i}\rbrack
\end{dmath}

The values for the correlation time and length are varied throughout the
experiments.

\subsection{Process Overview}
The simulation is initialized by creating the virtual environment (with
salinity and inundation values calculated for each patch using their
distance from the sea) and planting the initial populations or
mangroves.

During each step, a Poisson-distributed random time interval
\(\text{dt}\)with a mean of 1 day is generated as per \cite{SalmoJuanico2015}. Each
agent/mangrove grows in diameter according to the growth equation and
proportionally to\(\text{dt}\); that is, the change in diameter during a
step in the simulation is calculated
as\(\frac{\text{dD}}{\text{dt}}dt = dD\).

During the same step, the recruitment chance for each patch is
recalculated. A randomly selected mangrove may die from ``natural
causes'', with a probability dependent on its maturity. A randomly
selected patch may also recruit a new seedling dependent on its
recruitment chance.

Storms also occur as a Poisson process. Each mature tree, irrespective
of species, has a probability of dying equal to\(0.7 - kn\),
where\(k\)is a parameter of the simulation and\(n\)is the number of
neighboring trees within the focal tree's field of neighborhood. Smaller
plants have a 0.1 chance of dying.

\subsection{Design Concepts}

Storms and death are both modeled as natural Poisson processes with
varying rates depending on species and maturity. Seedling dispersal is
modeled as diffusion using spatiotemporal colored noise.

\subsubsection{Emergence and Adaptation}
The model can simulate the emergence of a naturally-selected population
of mangroves over several generations, affected by both the
fragmentation of the mangrove forest (plants are not uniformly
distributed) and the properties of the plants themselves. It will thus
be able to take into account the effects of having multiple species and
an arbitrary landscape.

The variations in the properties of each individual agent or plant, as
well as the tendency for plants that are more fit to survive longer and
have more offspring will simulate spatial selection within the
population.

\subsubsection{Interaction}
Plant agents interact with each other by competing for resources, which
affects growth, and thus also influences mortality rates. The more
neighbors around it a plant has, the slower it should grow, and thus the
greater the chances of it dying during each step. However, having more
neighbors also reduces the mortality rate of the tree during storms,
since having more neighbors distributes the force and damage inflicted
by the storm.
