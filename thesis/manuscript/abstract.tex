Mangrove forests along tropical coastlines frequently suffer severe damage due to storms. The mangrove trees are extremely sensitive to environmental stressors such as water salinity and tidal inundation. There is a need for accurate modelling of such systems in order to assist in the decision-making process of environmental conservationists. The authors propose an agent-based model for the prediction of the regenerative behaviour of mangrove stands consisting of the native species and the planted or non-native species in a fragmented habitat, with the use of spatio-temporal coloured noise to simulate stochastic seedling dispersal. It uses Salmo and Juanico's model for individual mangrove growth. This study aims to model accurately the growth of mangrove populations while experiencing said factors and subject to storm damage. Recovery of a population is determined in terms of the current total forest cover compared to the total forest cover before the storm occurred.

Stochastic experiments were carried out in a virtual analogue of Bangrin Marine Protected Area in Bani, Pangasinan with a population of native mangroves (\textit{Avicennia} and \textit{Sonneratia}) and a larger population of planted, non-native mangroves (\textit{Rhizophora}). Out of 1280 runs of various configurations, the native genera always fully recovered within 5.65 to  6.84 years. The planted genera only fully recovered 6 times out of 1280. Within 4.53 to 5.45 years, the planted mangroves were only able to regain around 66.16\% to 71.30\% of their total pre-storm cover.