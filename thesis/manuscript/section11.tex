In 2009, Typhoon "Chan-hom" caused heavy damage in the coastal woodlands of Lingayen Gulf in the Philippines \cite{Salmo2013}. Damage was dealt  heterogeneously across the mangrove stands, with stands between 11 and 18 years old suffering the greatest amount of damage. Salmo, Lovelock and Duke further noted that these may be near the maximum size attainable by mangroves in the region due to the high frequency of storms. This may also contribute to the low biomass reported for mangal forests in the Philippines. Field observations indicated that, in general, taller trees (or those with larger diameters) have a higher mortality rate during storms compared to their shorter neighbours. This is because they incur the greatest amount of damage from storm winds. 

Due to their "lower structural complexity and lower wind firmness", non-native mangrove plants have been observed to be "more vulnerable to typhoons compared to natural stands" \cite{Salmo2013}. It should be noted, however, that observed restoration trajectories of mangrove plantations indicate that after a number of years (around 50 years for \emph{Rhizophora}) they are able to match soil and vegetation characteristics of natural forests \cite{Salmo2011}.

There are several factors that may affect forest recovery post-disaster. After Hurricane Andrew in the United States, it was observed among fringe mangrove forests in Florida that the number of below-canopy seedlings that survive a storm can impact the recovery trajectory \cite{Baldwin2001}. 

Growth proceeds rapidly among young mangroves and slows down as the plants mature. Increased tree mortalities continue even 9 months or up to 2 years after a severe storm \cite{Salmo2013}. Salmo, Lovelock and Duke \cite{Salmo2011} also observed a "clear progression" in the organic matter, Nitrogen, Phosphorus, redox potential and temperature of the soil as mangroves matured. Changes in soil characteristics may have also occurred post-disaster, particularly an increase in soil nutrients, reduced organic matter, more anoxic soils and increased salinity and soil temperature \cite{Salmo2013}. Due to these factors, post-disaster forest structure may become significantly different from pre-disaster structure \cite{Baldwin2001}.