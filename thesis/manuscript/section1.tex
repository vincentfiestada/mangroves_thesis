Mangroves are several families of woody plants that live in areas where
land meets water in tropical and subtropical regions. They can be found
in salty and brackish waters, such as coastlines and swamps. Mangroves
are a highly diverse group of plants, each uniquely suited to their own
environment. Indeed, there is a significant amount of intra- and
inter-species genetic variability among mangroves \cite{biologyOfMangroves}.

They, along with many kinds of bacteria, fungi, plants, animals and
other organisms form the \emph{mangal} or mangrove ecosystem. Mangroves
are a vital part of these communities and the ecosystems formed by them.
They have several ecological benefits including serving as a habitat
fish, carbon sequestration, protection of shorelines, and nutrient
regulation \cite{roleOfMangroves}. Mangrove-associated flora and fauna include algae,
sea grasses, saltmarsh plants, sponges, nematodes, prawns, shrimp,
crabs, dragonflies, bees, ants, shipworms, and fish--to name a few \cite{biologyOfMangroves}.

Climate change has caused mangrove populations to decline significantly
and continuously in recent years. Coastal woodlands are disappearing the
fastest in developing countries, where close to 90\% of mangroves are
found \cite{kavanagh2007}. This endangers not only the mangroves, but a whole swath
of other organisms dependent on them for the survival of their
ecosystems. Thus, environmental conservation efforts that focus on mangroves also have the
potential of preserving the unique environmental services from coastal wetlands.

In some cases, it is necessary to arrange rehabilitation efforts in the interest of environmental conservation because of the fragility of mangal ecosystems. For instance, in the Philippines, the majority of mangrove stands may never reach full maturity because of continuous damage from severe typhoons \cite{Salmo2011}. This may require replanting efforts in order to assist in coastal forest recovery.