We shall now discuss the possible ecological models applicable to mangrove forests. In the Block Disturbance model by Hiebeler \cite{HiebelerFragmentedHabitat2011}, agents are placed in an environment consisting of a lattice of tiles or patches that individuals can occupy. Periodic disturbances cause the death of a large number of agents. A "block" or large region in the lattice is randomly dealt a high amount of damage, killing many of its residents at once. Higher mortality in a region arises as a result of random \emph{block disturbances}.

The Fragmented Habitat model is another important ecological model. It is also agent-based, with agents inhabiting a lattice of patches. The habitat is said to be fragmented due to the non-uniform distribution of organisms. This is influenced in part by the suitability of organisms to certain parts of the lattice as well as a fixed per capita death rate \cite{Hiebeler2000}. In the case of this model, higher mortality in a region arises as a result of lower suitability of agents to the patches in that region.

From the combination of these two concepts emerge a system in which disturbances do a non-uniform amount of damage to different regions of the simulated environment. This model can be used for storms (which may be modeled as a Poisson process) that occur randomly to damage the mangrove populations. Every time a storm occurs, an area in the mangrove forest is more heavily damaged than other areas. This randomness could be likened to the different paths a storm or typhoon can take while passing through the coastal woodlands. 

Aside from damage from disturbances, there is a steady backdrop of mortality among the population. Certain agents may be more suitable to certain regions of the virtual environment, resulting in lower mortality rates in such situations. These two models form the basis of how mortality and survivability are realized in the proposed model.