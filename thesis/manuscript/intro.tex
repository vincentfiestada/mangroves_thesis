\chapter{Introduction}
\label{cha:intro}

\section{Background}
Mangroves are trees and shrubs that live in the coastal intertidal zone in tropical and subtropical regions. The root system of mangrove forests serve as a habitat to various aquatic organisms. Mangroves are also used in small-scale industries such as in flavouring agents, textiles, housing and more.

Mangroves live in highly saline, anoxic environments. Their seedlings can float and take root once it finds ground. Some species are \emph{viviparous} and have seeds that develop while still attached to the parent plant, then break off afterwards. Some mangroves have been known to live for more than 100 years. However, most trees will never reach full maturity because of various factors affecting mangrove growth and mortality. These include tidal inundation, salinity, and competition between individuals.

Coastal woodlands are also very vulnerable to storms. In 2009, Typhoon "Emong", known internationally as typhoon "Chan-hom", dealt major damage to the mangrove forests of the Philippines. While there had been replanting efforts in some areas, the locals opted to plant non-native \emph{Rhizophora} instead of \emph{Avicennia and Sonneratia}, which are native to the region. The decisions made regarding those efforts have been put into question due to the suitability of certain species to the site and the effects on recovery speed.

\section{Problem Statement}
The aim of this study is to develop a model that accurately simulates the regenerative behaviour of a multi-specific mangrove forest in a fragmented habitat influenced by tidal inundation, salinity, inter- and intra-species competition, and storms.

The agent-based modelling paradigm is used in order to simulate this system. Such a paradigm can be used in cases of complex systems for which no straightforward systems of equations exist that can model the population dynamics as empirically observed. The authors improve upon a model for individual growth developed by Salmo \& Juanico \cite{SalmoJuanico2015} and an agent-based implementation by Ang \& Mariano \cite{mangrovesAngMariano}.

\section{Significance of the Study}
There is a need for a stochastic model that can predict the regenerative behaviour of single and multi-species mangrove forests in a fragmented habitat. Such a model can help inform the decision-making process of environmental conservation efforts. 

As previously stated, mangroves are incredibly important components of mangal ecosystems. They serve as habitat and nursing grounds for various aquatic organisms. They are also used in various small-scale industries. However, mangrove forests routinely sustain heavy damage from typhoons. Rehabilitation efforts can play a key role in ensuring that they recover as quickly as possible. This model can be used to evaluate hypothetical rehabilitation programs for coastal woodlands in an inexpensive manner. It can assist biologists and environmentalists in choosing which genera of mangroves to plant and how that choice will affect the speed of recovery.